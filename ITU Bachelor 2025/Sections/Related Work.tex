\section{Related Work}
\subsection{Using LLMs to generate documentation}
\label{sec:related work LLM}
Research on LLMs ability to generate documentation from source code suggest that there is as potential for automating the task of documentation.
\\ \\
Khan and Uddin, \cite{10.1145/3551349.3559548} used a Chat GPT-3 based model called Codex, which is trained on natural language and code, to generate code documentations given source code from the CodeSearchNet dataset, covering six languages (Java, Python, PHP, Go, JavaScript, Ruby). Between the generated documentation and the original documentation from the dataset, several quantitative metrics and a qualitative comparison was used to evaluated the result. Showing that source code documentation generated by Codex had satisfactory scores in the quantitative metrics and even better understandability and extra information were added by Codex in the qualitative analysis.
\\ \\
Dvivedi et al. \cite{dvivedi2024comparativeanalysislargelanguage} conducted an analysis on how well LLMs generated source code documentation given the code base, compared to the original code documentation. The model they used were, at the time, the most popular closed- and open-source LLMs, with sizes from 15 billion parameters (Starchat) to 1.76 trillion parameters. They looked at 5 objective metrics: Accuracy, Completeness, Relevance, Understandability and Readability. Concluding that:
\begin{quote}
    \textit{``with the exception of Starchat, all LLMs demonstrated either equivalent or superior performance compared to the original documentation, highlighting their potential for automating documentation tasks."}
\end{quote}

\subsection{Autonomous Code Documentation Agents}
\label{sec: autonomus code documentaioin agents}
Looking at research on prototypes which tries to automate the task of generating documentation from source code optained from a GitHub repository, we see that the use of LLM-based Agents is a typical approach.
\\ \\
Yang et al.\cite{yang2025docagent} propose DocAgent, a multi-agent collaborative system, which generate documentation for a whole repository by leveraging the Chat GPT API. Working in two step: Firstly the Navigator makes a optimal process order, by examine the dependencies in the Repository. Secondly their multi agent system generates documentation for the whole repository.
\\ \\
Luo et al.\cite{luo2024repoagent} have made RepoAGENT which uses Chat GPTs API to generate documentation files, which analyses the changes made in the commit, and updates the documentation files. They  trigger this process by utilizing the git pre-commit hook, thus managing to integrate the generation into a GitHub reposiotry.
\\ \\
With confidence that various LLM is able to generate method level documentation, see \ref{sec:related work LLM}, this paper will contribute with a proposal to an autonomous code documentation agent prototype which will combine the two approaches mentioned above \ref{sec: autonomus code documentaioin agents} by both offering source code documentation and the integration into a GitHub repository. But also be in contrast by offering the agent as a free service without the need for a subscription to a cloud hosted AI model, like Chat GPT API.

