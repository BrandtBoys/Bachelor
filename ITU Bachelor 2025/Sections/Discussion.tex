\section{Discussion}
In this section we will touch back on the stated sub-research-questions and discuss what value the applied method deliver when trying to answer these, and finally we discuss our initial research question and look at the bigger picture by looking at what initially motivated the paper and how this paper contributes in that regard.
\subsection{Q1 Discussion}
\label{sec:DiscussionQ1}
"\textit{\subquestionI}"
During development of DocTide several challenges and limitations was encountered in the quest to achieve the behavior of an autonomous agent on the hardware which the GitHub Runner provided. This section will discuss how these hardware limitation influenced the design of DocTide when trying to follow the proposed design of an LLM-based autonomous Agent from Wang et al.\cite{wang2024survey}.
\subsubsection*{Letting the LLM do it all}
The first approach was to find an LLM which fitted the hardware limitation, and as one of the first suggestions we met on Ollama was the llama 3.2 with a size of 2 gb in memory. Our intuition was that this size wouldn't exhorts the 14 gb RAM (\Cref{sec:githubRunner}), and therefor should be able to run fast. And when provided the whole file, on max 20 line, it did its job fairly good. But when we got our real world evaluation framework (\Cref{sec:DocTideLabs}) set up, it medially exposed the small LLMs limits when having to deal with file sizes of upwards to 600 lines (\Cref{appendix:A}). It is not possible to have a small LLM understand the whole context and write the whole file from scratch, hoping it have made sensible function level documentation, and not broken the code. 
\subsubsection*{Identifying the modified functions to document}
But since we had to stick with the small LLM, a way to offload the LLMs work, was to analyze the context on its behalf using tree-sitter and diff\_lib. If we should have stayed more true to the proposed agent design by Wang et al.\cite{wang2024survey} we should probably have tried to make a \textcolor{red}{LLM-drive planing module,|action?} which have identified and served the modified functions to an LLM-driven action which then could have generate the function level documentation. But since our level of trust in the small LLM to understand such large files had decreased, we opted for gaining more trust to the system by using static analysis. But this was not the only place we had to cut down the agents autonomy and using the content knowledge tree-sitter provides.
\subsubsection*{Write documentation back into the source code}
The test environment also exposed how the number of modified functions increased the actions execution time.
Execution times came upwards to 3 hours on merge commits, where multiply files was commit at once, and when investigating which process in the action that took so much time, and it was obvious that it was when the LLM had to write whole files and try to add the function level documentation the correct place. The approach with AST was further used to figure out where exactly a function level comment should be place given a specific modified function, relieving the LLMs task to only generate the function level documentation, provided the code of the modified function, and pass it to the insert functionality which takes the calculated byte offset to where the documentation should be, from the context of the AST, and insert it into the source code. The decreased the execution time significantly, but again we opted to go with an static analysis approach, rather than making a new LLM-based action which only purpose was to figure out where the most appropriate location for the documentation would be. The approach chosen gave us the most trust in that the agent would place the documentation at the right location, but we in turn decreased its autonomous abilities. And in fact, letting an LLM asses where it is appropriate to place a function-level documentation would make it more portable and not limited only identify where a comment should be placed given the supported programming language or level of documentation.
\subsubsection*{Ci integration}


\subsection{Q2 Discussion}
\label{sec:DiscussionQ2}
"\textit{\subquestionII}"
A semantic similarity score should, opposite to just looking at character comparison, give a quantitative measure for how much of the semantic meaning is present between two texts. This is why we chose to use the semantic similarity model from Sentence Transformers to compare the similarity between function-level documentation created by a human developer and by DocTide, to get a understanding of how similar to what the human developers intent to explain in a function-level documentation DocTide manages to generate. And in our qualitative analysis of the generated semantic similarity score(\Cref{sec:identifying thresholds}) we found this models scores to be a good indicate for this measurement. This measurement does not cover all attributes of documentation and does therefor not cover the all the ways to asses to what level DocTide is able to produce function-level documentation compared to that of a human developer, but it gives an indication on one of the more humane attributes by looking at what level DocTide captures the same intent with the documentation as the developer does.
Reflecting upon why the score was as it was, when looking at some of the data there were a correlation between poorly scored documentations and the developer documentation containing context which was derived from other files or how the function was used and not only its implementation. 
This could hypothesize that introducing more context to DocTide would improve its semantic scores, this hypothesis could be further motivated by having a developer create function-level documentation given only the source code of the function, operating under the same restrictions as DocTide, and see if the developers semantic score would follow the same distribution, indicating that it is the lack of context that is a factor for the level of similarity or if the developer scored higher, indicating that human intuition is a driving factor.
\subsection{Q3 Discussion}
\label{sec:DiscussionQ3}
"\textit{\subquestionIII}"
The results of the experiment as described in \cref{sec:suc_results} shows a \textit{'success\_rate'} for DocTide at \textit{68.99\%}, meaning that \textit{68.99\%} of the time the DocTide agent has attempted to generate documentation, it has resulted in documentation following the format described in \cref{sec:method} which has been inserted back into the code base. 
\subsubsection*{Current documentation coverage}
To discuss the implications of this, the python tool \textit{'docstr-coverage'}\footnote{\url{https://pypi.org/project/docstr-coverage/}} is ran on the same repository as the experiment was executed. This tool scans the entire codebase of the repository, identifies possible locations for method-level documentation, and meassures this coverage of these. As seen in \cref{lst:doc_percent_flask}, the results of this shows that the repository has method level documentation coverage of \textit{38.7\%}, implying that if the same project was ran with only DocTide handling documentation, the project would have a higher coverage of method level documentation than its current state. Though as found in \cref{sec:sem_results} not with the same level of information conveyed in each comment.

\begin{lstlisting}[language=sh, label={lst:doc_percent_flask},      caption= Ratio of method level documentation in flask repository]
    Needed: 8193  -  Found: 3169  -  Missing: 5024
    Total coverage: 38.7%  -  Grade: Not good
\end{lstlisting}

\subsubsection*{Format failures}
When looking at random samples collected from the 31.01\% of the dataset where the documentation generated does not follow the format described in \cref{sec:method}, two common reasons stands out as recurring reasons to format failure: the agent either misses the set of quotes actively 'closing' the comment, or the agent returns a copy of the method for which it is generating documentation as a prefix or suffix to the documentation itself. 

DocTide follows a very conservative approach where the attempt at generating documentation for a given function, is aborted when documentation not following the correct format is produced. After identifying the pattern of two common recurring reasons for format failure, it is hypothesized that integrating some form of internal feedback mechanism biased towards handling these exact cases could lead to an improvement in the 'success\_rate' metric. Following \cite{wang2024survey} one consideration could be to implement \textit{model feedback} where internally an LLM is used instead of DocTides static format check, to decide whether the generated documentation follows correct format and to handling \textit{re-prompting} in the case that it does not. Similar to what is discussed in \cref{sec:DiscussionQ1}, this is limited by the capabilities of the LLM leading to a tradeoff between autonomy and trust.

\subsection{Overall discussion}
Remember to use motivation from introduction

Building and integrating a LLM-based autonomous agent has exposed a clear bias in our developing approach, which was evident in the solutions we resolved to when facing the challenges described in \Cref{sec:DiscussionQ1}. With a mindset of software developers working towards a working product, the level of autonomy was kept to a minimum to ensure predictable behavior. This notion of a balance between trust and autonomy was a great help to articulate our motivation for our decisions, which we had not meet in research, but we find to be true and relevant with more and more AI-based tools being integrated in the development life cycle, Developers have to make up how much they trust given systems to act autonomously in their code, especial when it comes to code generation. But our "conservative" approach did make us wonder if we had removed so much autonomy to the point where DocTide could not be considered as an autonomous agent. How ever we believe DocTide to rightfully be considered autonomous since it stills senses it environment 


But with that said, the use of the proposed design of LLM-based autonomous agents by Wang et al.\cite{wang2024survey} provided great direction and considerations on how to extend the autonomy of the agent and practical approaches to how to achieve that. Which is why we see many untried approaches which we believe could increase the performance and autonomy of DocTide. 



Level of autonomy: We set of with a medium level of autonomy, but with the limited size/capabilities of the LLM model, and the vitality of not injecting working code into the code base, we limited the level of autonomy to a minimum. With more careful approaches, this level could be rissen, by letting the LLM asses its own output, and make it evaluate if it is a satisfactory documentation of the code.

\subsection{Future work}
\begin{itemize}
    \item more intent 
        \item persistent memory with config files, both users of DocTide can use to define some overarcing concepts, and to store learnings made from modifying in PRs
        \item 
\end{itemize}