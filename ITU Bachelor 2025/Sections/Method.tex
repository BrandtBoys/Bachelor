\newlist{qlist}{itemize}{1}
\setlist[qlist]{label=\textbf{Q1:}}
\newcommand\itemb{\item[\textbf{Q2:}]}
\newcommand\itemc{\item[\textbf{Q3:}]}

\section{Methodology}
\label{sec:method}
The term `documentation' will for the remainder of this paper refer to the notion of method-level documentation.
\\ \\
As stated in the introduction the contribution of this paper is to investigate the following:
\begin{quote}
    \researchQuestion
\end{quote}
The methodology used to do so, is to first identify the three sub-questions. The questions is presented alongside their motivation in the following:

\begin{qlist}
    \item \subquestionI\textbf{:}
    Looking at what considerations one should take into account when designing an autonomous agent under the restrictions of integrating said agent into a github environment, and how challenges faced influences the proposed design.
    \itemb \subquestionII\textbf{:}
    The capabilities to generating documentation of DocTide is compared to that of a human developer, to understand how well DocTide as an autonomous agent would replace a human developer in the task of generating documentation.
    \itemc \subquestionIII\textbf{:}
    Since the core functionality of DocTide is to generate documentation to be injected directly into the code base, it is necessary that both the syntax is correct such that the code can compile and that only documentation is created, to not endanger the code of breaking or change behavior.
\end{qlist}

\noindent
To investigate \textbf{Q1}, a prototype of an autonomous agent, DocTide, is developed, following the design described in section \ref{sec:BackgroundAgentDesign}. The results of this is presented in section \ref{sec:DesignDocTide}, and discussed in section \ref{sec:DiscussionQ1}. The \textit{'human similarity metric'} and \textit{'success metric'} described in the \textit{LLM-based autonomous agent evaluation} section in Wang et al.\cite{wang2024survey} was identified as appropriate measures to address \textbf{Q2} and \textbf{Q3} respectively. These metrics are collected by running an experiment described in section \ref{sec:exp}, using a reusable evaluation framework, DocTide Labs (Section \ref{sec:DocTideLabs}). DocTide Labs was designed to facilitate the seamless integration of DocTide into a repository and to enable the systematic collection of data for the specified metrics.

The \textit{`human similarity metric'} is defined as the similarity in meaning between documentation DocTide creates and original documentation present in the repository. This is scored using the CossEncoder from sentence transformers\footnote{\url{https://sbert.net/examples/cross_encoder/applications/README.html}} to get a semantic similarity score between existing documentation and documentation created by DocTide. To understand the scoring produced, the threshold is identified qualitatively between three buckets identified as follows:
\begin{itemize}
    \item \textbf{Low similarity:} Agent generated documentation fails to convey any of the information provided in the original documentation.
    \item \textbf{Medium similarity:} Agent generated documentation convey some correct information reflecting parts of the original documentation.
    \item \textbf{High similarity:} Agent generated documentation correctly conveys the majority of information provided in the original documentation.
\end{itemize}
The results of this is presented in section \ref{sec:sem_results}, and discussed in section \ref{sec:DiscussionQ2}.
\\ \\
Derived from the motivation of \textbf{Q3} the \textit{'success metric'} is defined as the ratio between the number off successful formatted documentations generated by DocTide and the total number of attempted documentations. The results of this is presented in section \ref{sec:suc_results}, and discussed in section \ref{sec:DiscussionQ3}.

\subsection{AI usage disclaimer}
\textit{During the development of the code base for this project, openAI's ChatGPT 4o has been used as a tool for enhancing our software development. It has been used to understand the used frameworks and technologies as well as to understand error messages, but has not been used to generate code verbatim for the code base.
}