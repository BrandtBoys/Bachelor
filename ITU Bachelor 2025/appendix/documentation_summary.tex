\begin{tabular}{rlllllll}
\toprule
Semantic-Score & Code & Original-Comment & Agent-Comment & Filename & Agent-Commit & Score-Bin & Label \\
\midrule
0.475059 & def save_session & """This is called at the end of each request, after generating
        a response, before removing the request context. It is skipped
        if :meth:`is_null_session` returns ``True``.
        """ & """
Saves a session.

This method is intended to be overridden by subclasses. It takes in the Flask application,
the session object, and the response object as parameters. The implementation of this
method should be provided by the subclass.

Parameters:
app (Flask): The Flask application instance.
session (SessionMixin): The session object.
response (Response): The response object.

Returns:
None

Raises:
NotImplementedError: This method is intended to be overridden and should not be called directly.
""" & src/flask/sessions.py & 76e12bb4372692bef31a46a90333ee3cde90c7f0 & (0.428, 0.476] & L \\
0.428235 & def env & """What environment the app is running in. This maps to the :data:`ENV` config
        key.

        **Do not enable development when deploying in production.**

        Default: ``'production'``

        .. deprecated:: 2.2
            Will be removed in Flask 2.3.
        """ & """
Deprecation Warning: `env` method is deprecated and will be removed in Flask 2.3.
Use `debug` attribute instead.

Args:
    value (str): The environment variable to set.

Returns:
    None
""" & src/flask/app.py & e27319e633c170ca8c59ba76332614b67db41f23 & (0.428, 0.476] & L \\
0.512154 & def templates_auto_reload & """Reload templates when they are changed. Used by
        :meth:`create_jinja_environment`. It is enabled by default in debug mode.

        .. deprecated:: 2.2
            Will be removed in Flask 2.3. Use ``app.config["TEMPLATES_AUTO_RELOAD"]``
            instead.

        .. versionadded:: 1.0
            This property was added but the underlying config and behavior
            already existed.
        """ & """
Returns the value of `TEMPLATES_AUTO_RELOAD` from the application configuration.

If `TEMPLATES_AUTO_RELOAD` is set, its value is returned. Otherwise, the value of `debug` is used as a fallback.

Deprecated since Flask 2.3 in favor of using `TEMPLATES_AUTO_RELOAD` in `app.config`.

Args:
    None

Returns:
    bool: The value of `TEMPLATES_AUTO_RELOAD` or `debug` if not set.
""" & src/flask/app.py & e27319e633c170ca8c59ba76332614b67db41f23 & (0.476, 0.525] & L \\
0.516763 & def session_transaction & """When used in combination with a ``with`` statement this opens a
        session transaction.  This can be used to modify the session that
        the test client uses.  Once the ``with`` block is left the session is
        stored back.

        ::

            with client.session_transaction() as session:
                session['value'] = 42

        Internally this is implemented by going through a temporary test
        request context and since session handling could depend on
        request variables this function accepts the same arguments as
        :meth:`~flask.Flask.test_request_context` which are directly
        passed through.
        """ & """
Yield a session object for the current test request context.

This function is used to create and manage sessions for testing purposes.
It checks if cookies are enabled, sets up the WSGI context, opens a new session,
and saves it after use. If the session backend fails to open a session, a
RuntimeError is raised.

Args:
    *args: Variable length argument list containing any arguments passed to the test request context.
    **kwargs: Keyworded arguments for the test request context.

Returns:
    A generator yielding SessionMixin objects.
""" & src/flask/testing.py & 223f05ba6a9f2fdf4bdbbb8072a3526458363344 & (0.476, 0.525] & M \\
0.568346 & def wsgi_app & """The actual WSGI application. This is not implemented in
        :meth:`__call__` so that middlewares can be applied without
        losing a reference to the app object. Instead of doing this::

            app = MyMiddleware(app)

        It's a better idea to do this instead::

            app.wsgi_app = MyMiddleware(app.wsgi_app)

        Then you still have the original application object around and
        can continue to call methods on it.

        .. versionchanged:: 0.7
            Teardown events for the request and app contexts are called
            even if an unhandled error occurs. Other events may not be
            called depending on when an error occurs during dispatch.
            See :ref:`callbacks-and-errors`.

        :param environ: A WSGI environment.
        :param start_response: A callable accepting a status code,
            a list of headers, and an optional exception context to
            start the response.
        """ & """
WSGI Application Function

This function serves as the entry point for the WSGI application. It takes in an environment dictionary and a start response callable, 
and returns any response object generated by the application.

The function first creates a request context using `self.request_context(environ)`. It then attempts to execute the full dispatch of the request,
handling any exceptions that may occur during this process. If an exception is caught, it will be handled and propagated up the call stack.
Finally, the response object is returned to the caller.

Note: This function should not be called directly by users of the application. Instead, it should be used as part of a larger WSGI server or framework.
""" & src/flask/sansio/app.py & b989889728c3e5967356d041d67a15949fd3bade & (0.525, 0.573] & L \\
0.555169 & def static_url_path & """The URL prefix that the static route will be accessible from.

        If it was not configured during init, it is derived from
        :attr:`static_folder`.
        """ & """
Returns a static URL path.

This method takes a string or None as input and returns the URL path after removing any trailing slashes. If the input is None, it sets the internal `_static_url_path` attribute to None.

Args:
    value (str | None): The URL path to be processed.

Returns:
    None
""" & src/flask/sansio/scaffold.py & b989889728c3e5967356d041d67a15949fd3bade & (0.525, 0.573] & L \\
0.554580 & def process_response & """Can be overridden in order to modify the response object
        before it's sent to the WSGI server.  By default this will
        call all the :meth:`after_request` decorated functions.

        .. versionchanged:: 0.5
           As of Flask 0.5 the functions registered for after request
           execution are called in reverse order of registration.

        :param response: a :attr:`response_class` object.
        :return: a new response object or the same, has to be an
                 instance of :attr:`response_class`.
        """ & """
Processes the given response by executing any after-request functions and saving the session.

Args:
    response (Response): The response to be processed.

Returns:
    Response: The processed response.
""" & src/flask/sansio/app.py & b989889728c3e5967356d041d67a15949fd3bade & (0.525, 0.573] & M \\
0.536651 & def env & """What environment the app is running in. This maps to the :data:`ENV` config
        key.

        **Do not enable development when deploying in production.**

        Default: ``'production'``

        .. deprecated:: 2.2
            Will be removed in Flask 2.3.
        """ & """
Returns the environment variable as a string.

Deprecation Warning: This method is deprecated and will be removed in Flask 2.3.
Use `app.debug` instead.

Args:
    None

Returns:
    str: The environment variable value.

Raises:
    DeprecationWarning: If the 'app.env' method is called.
""" & src/flask/app.py & e27319e633c170ca8c59ba76332614b67db41f23 & (0.525, 0.573] & M \\
0.618780 & def finalize_request & """Given the return value from a view function this finalizes
        the request by converting it into a response and invoking the
        postprocessing functions.  This is invoked for both normal
        request dispatching as well as error handlers.

        Because this means that it might be called as a result of a
        failure a special safe mode is available which can be enabled
        with the `from_error_handler` flag.  If enabled, failures in
        response processing will be logged and otherwise ignored.

        :internal:
        """ & """
Finalizes a request by processing the response and sending a signal to indicate that the request has finished.

Args:
    rv (ft.ResponseReturnValue | HTTPException): The response value or exception to be finalized.
    from_error_handler (bool, optional): Whether this is being called from an error handler. Defaults to False.

Returns:
    Response: The finalized response object.

Raises:
    Exception: If the request finalizing fails and `from_error_handler` is False.
""" & src/flask/sansio/app.py & b989889728c3e5967356d041d67a15949fd3bade & (0.573, 0.621] & M \\
0.604298 & def request_context & """Create a :class:`~flask.ctx.RequestContext` representing a
        WSGI environment. Use a ``with`` block to push the context,
        which will make :data:`request` point at this request.

        See :doc:`/reqcontext`.

        Typically you should not call this from your own code. A request
        context is automatically pushed by the :meth:`wsgi_app` when
        handling a request. Use :meth:`test_request_context` to create
        an environment and context instead of this method.

        :param environ: a WSGI environment
        """ & """
Returns a new instance of RequestContext with the given environment.

Args:
    environ (dict): The current HTTP environment.

Returns:
    RequestContext: A new instance of RequestContext.
""" & src/flask/sansio/app.py & b989889728c3e5967356d041d67a15949fd3bade & (0.573, 0.621] & M \\
0.596418 & def preprocess_request & """Called before the request is dispatched. Calls
        :attr:`url_value_preprocessors` registered with the app and the
        current blueprint (if any). Then calls :attr:`before_request_funcs`
        registered with the app and the blueprint.

        If any :meth:`before_request` handler returns a non-None value, the
        value is handled as if it was the return value from the view, and
        further request handling is stopped.
        """ & """
Preprocesses the request by applying URL value preprocessors and before request functions.

This method iterates over the blueprint names in reverse order, applying any URL value preprocessors to each one.
It then iterates over the same list again, applying any before request functions to each one. If a function returns
a non-None value, it is returned immediately. Otherwise, None is returned at the end.

Args:
    self: The instance of the class this method belongs to.

Returns:
    ft.ResponseReturnValue | None: The result of the preprocess request, or None if no functions return a value.
""" & src/flask/app.py & b36820535d51b007c2551e9b65752754c632f573 & (0.573, 0.621] & H \\
0.618487 & def create_jinja_environment & """Create the Jinja environment based on :attr:`jinja_options`
        and the various Jinja-related methods of the app. Changing
        :attr:`jinja_options` after this will have no effect. Also adds
        Flask-related globals and filters to the environment.

        .. versionchanged:: 0.11
           ``Environment.auto_reload`` set in accordance with
           ``TEMPLATES_AUTO_RELOAD`` configuration option.

        .. versionadded:: 0.5
        """ & """
Creates a Jinja environment with custom options and updates its globals.

This method creates a new Jinja environment based on the provided options.
It also updates the environment's globals dictionary to include necessary functions
and variables for use in templates.

Args:
    self: The object instance that owns this method.

Returns:
    Environment: A newly created Jinja environment with custom options and updated globals.
""" & src/flask/sansio/app.py & b989889728c3e5967356d041d67a15949fd3bade & (0.573, 0.621] & H \\
0.601796 & def _validate_key & """The ``--key`` option must be specified when ``--cert`` is a file.
    Modifies the ``cert`` param to be a ``(cert, key)`` pair if needed.
    """ & """
Validate the key for a given certificate.

This function checks if the provided key is valid based on the type of certificate used.
It raises an error if the key is not required or if it's used with an invalid certificate type.

Args:
    ctx (click.Context): The context object containing the command-line arguments.
    param (click.Parameter): The parameter being validated.
    value: The value to be validated.

Returns:
    value: The validated key value.

Raises:
    click.BadParameter: If the key is not required or if it's used with an invalid certificate type.
""" & src/flask/cli.py & b36820535d51b007c2551e9b65752754c632f573 & (0.573, 0.621] & L \\
0.614353 & def load_app & """Loads the Flask app (if not yet loaded) and returns it.  Calling
        this multiple times will just result in the already loaded app to
        be returned.
        """ & """
Loads a Flask application instance.

This method attempts to load an existing Flask application from the `self._loaded_app` attribute,
or creates a new one if none exists. It also sets the debug flag of the loaded application
if `self.set_debug_flag` is True.

If no application can be found, it raises a `NoAppException`.

Returns:
    Flask: The loaded or created Flask application instance.
""" & src/flask/cli.py & b36820535d51b007c2551e9b65752754c632f573 & (0.573, 0.621] & H \\
0.593989 & def handle_http_exception & """Handles an HTTP exception.  By default this will invoke the
        registered error handlers and fall back to returning the
        exception as response.

        .. versionchanged:: 1.0.3
            ``RoutingException``, used internally for actions such as
             slash redirects during routing, is not passed to error
             handlers.

        .. versionchanged:: 1.0
            Exceptions are looked up by code *and* by MRO, so
            ``HTTPException`` subclasses can be handled with a catch-all
            handler for the base ``HTTPException``.

        .. versionadded:: 0.3
        """ & """
Handles HTTP exceptions by checking their type and returning them accordingly.

If the exception does not have an error code (i.e., it's a ProxyException), 
it will be returned unchanged as an error. If it's a RoutingException, 
it will also be returned without modification. Otherwise, it will be 
passed to the error handler function to determine its response.

Args:
    e (HTTPException): The HTTP exception to handle.
Returns:
    HTTPException | ft.ResponseReturnValue: The handled exception or its response.
""" & src/flask/app.py & 3d2e1480774a331d776db25f5ec617329be02cc7 & (0.573, 0.621] & H \\
0.655802 & def open_instance_resource & """Opens a resource from the application's instance folder
        (:attr:`instance_path`).  Otherwise works like
        :meth:`open_resource`.  Instance resources can also be opened for
        writing.

        :param resource: the name of the resource.  To access resources within
                         subfolders use forward slashes as separator.
        :param mode: resource file opening mode, default is 'rb'.
        """ & """
Opens an instance resource file.

Args:
    - `resource` (str): The path to the resource file.
    - `mode` (str, optional): The mode in which to open the file. Defaults to "rb".

Returns:
    A file object opened at the specified location with the given mode.

Raises:
    FileNotFoundError: If the instance_path does not exist or the resource is not found.
""" & src/flask/sansio/app.py & b989889728c3e5967356d041d67a15949fd3bade & (0.621, 0.669] & H \\
0.627122 & def with_appcontext & """Wraps a callback so that it's guaranteed to be executed with the
    script's application context.

    Custom commands (and their options) registered under ``app.cli`` or
    ``blueprint.cli`` will always have an app context available, this
    decorator is not required in that case.

    .. versionchanged:: 2.2
        The app context is active for subcommands as well as the
        decorated callback. The app context is always available to
        ``app.cli`` command and parameter callbacks.
    """ & """
Decorates a function to run with the application context.

This decorator is used to ensure that functions decorated with it are executed within
the application's context. This can be useful for tasks such as database operations,
file I/O, or any other operation that requires access to the current application state.

The `with_appcontext` function takes a function `f` as an argument and returns the result of
invoking `f` with the application context. If the application context is not already set,
it will load the application from the `ScriptInfo` object associated with the click context.

Args:
    f (function): The function to be decorated.

Returns:
    function: The original function, wrapped in a decorator that runs it with the application context.
""" & src/flask/cli.py & 64712f525beb09fcd8ae56052b7c3081d35568ea & (0.621, 0.669] & M \\
0.652406 & def iter_blueprints & """Iterates over all blueprints by the order they were registered.

        .. versionadded:: 0.11
        """ & """
Returns an iterator over the blueprint values.

This method provides a view of all blueprints in the system, allowing for efficient iteration and access to their attributes. The returned iterator is a `ValuesView` object, which supports various methods for filtering and manipulating the results.

Args:
    None

Returns:
    t.ValuesView[Blueprint]: An iterator over the blueprint values.
""" & src/flask/sansio/app.py & b989889728c3e5967356d041d67a15949fd3bade & (0.621, 0.669] & H \\
0.625945 & def json_encoder & """Blueprint-local JSON encoder class to use. Set to ``None`` to use the app's.

        .. deprecated:: 2.2
             Will be removed in Flask 2.3. Customize
             :attr:`json_provider_class` instead.

        .. versionadded:: 0.10
        """ & """
Returns the JSON encoder class, deprecation warning if applicable.

This function is deprecated and will be removed in Flask 2.3. It's recommended to customize 'app.json_provider_class' or 'app.json' instead.

Args:
    None

Returns:
    t.Union[t.Type[json.JSONEncoder], None]: The JSON encoder class or None.
""" & src/flask/blueprints.py & e27319e633c170ca8c59ba76332614b67db41f23 & (0.621, 0.669] & H \\
0.693479 & def test_jsonify_uuid_types & """Test jsonify with uuid.UUID types""" & """
Tests the JSONification of UUID types.

This function tests that a UUID object can be successfully serialized to JSON and deserialized back into a UUID object.

Parameters:
app (Flask application): The Flask application instance.
client (Flask client): The Flask client instance.

Returns:
None
""" & tests/test_json.py & 30e153f81ce882c3b184a9429a248c6f89af6be4 & (0.669, 0.717] & H \\
0.698150 & def get_cookie_httponly & """Returns True if the session cookie should be httponly.  This
        currently just returns the value of the ``SESSION_COOKIE_HTTPONLY``
        config var.
        """ & """
Returns whether the session cookie is set to be HTTP-only.

Args:
    app (Flask): The application instance.

Returns:
    bool: True if the session cookie is HTTP-only, False otherwise.
""" & src/flask/sessions.py & b36820535d51b007c2551e9b65752754c632f573 & (0.669, 0.717] & M \\
0.745833 & def endpoint & """Decorate a view function to register it for the given
        endpoint. Used if a rule is added without a ``view_func`` with
        :meth:`add_url_rule`.

        .. code-block:: python

            app.add_url_rule("/ex", endpoint="example")

            @app.endpoint("example")
            def example():
                ...

        :param endpoint: The endpoint name to associate with the view
            function.
        """ & """
Endpoint Decorator Function

This function is a decorator that registers an endpoint with the provided endpoint string.
It takes in a function `f` and returns a new function that wraps the original function, 
registering it as a view function for the specified endpoint.

Args:
    endpoint (str): The endpoint to register the function under.
    f (Callable[[F], F]): The function to be registered as a view function.

Returns:
    Callable[[F], F]: A new function that wraps the original function and registers it as a view function.
""" & src/flask/sansio/scaffold.py & b989889728c3e5967356d041d67a15949fd3bade & (0.717, 0.766] & H \\
0.743081 & def tag & """Convert the value to a valid JSON type and add the tag structure
        around it.""" & """
Converts a given value to JSON format and returns it as a dictionary.

Args:
    value (t.Any): The value to be converted to JSON format.

Returns:
    dict[str, t.Any]: A dictionary containing the key-value pair where the key is 'tag' and the value is the JSON representation of the input value.
""" & src/flask/json/tag.py & b36820535d51b007c2551e9b65752754c632f573 & (0.717, 0.766] & H \\
\bottomrule
\end{tabular}
