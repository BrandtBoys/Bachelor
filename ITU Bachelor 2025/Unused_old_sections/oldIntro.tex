\subsection{\textcolor{red}{Automation in software development}}

The notion of automation is not new in the software development realm. As a part of the ever growing expectancy to the productivity in developing bigger and more complex system, the need for automation of processes in the software life-cycle was already seen as the necessary step to take back in 1985 where G. F. Hoffnagle and W. E. Beregi stated \textit{"Demand for reliable software systems is stressing software production capability, and automation is seen as a practical approach to increasing productivity and quality"}.


To ensure a 

Continuous integration is a widely used practices in software development, implemented as a process \textit{"which is typically automatically triggered and comprises inter-connected steps such as compiling code, running unit and acceptance tests, validating code coverage, checking compliance with coding standards, and building deployment packages."}\cite{FITZGERALD2017176}. 