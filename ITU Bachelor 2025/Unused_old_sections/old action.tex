
, the action used to fulfill the step in the plan and the \textit{impact} these action has. The full \textit{action space} is presented in \Cref{lst:plan_seq}
\\
The plan for which functions DocTide should generate document ion for and where in the code this function documentation should be inserted, is based on an static analysis of the modified files in the diff\_files list. Firstly the file type is analyzed to decide if the system supports the type, this is based on both the programming languages tree-sitters support, and business requirements, like yml, xml, txt files is not necessary to investigate further. Now tree-sitter is used to create a abstract syntax tree of the modified file, this makes it possible to assess any support programming language and identify modified functions, by looking at which lines in the code has been modified and if those lines overlaps with a function block. It then retrieves both the code of the whole function and the start point from where the function level documentation should be placed, which differs for different programming languages, e.g. in Python the function level documentation should be placed below the function definition, and in C\#, JS or Java the function level documentation should be above the function definition. The planing module now sets the parameters used in the prompt and invokes the LLM with the pre-defined prompt. The generated function documentation is then validated, and the feedback of this validation decides wether the attempted function documentation is committed or if its discarded.