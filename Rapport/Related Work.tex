\section{Related Work}
\subsection{Using LLMs to generate documentation}
\label{sec:related work LLM}
Research on LLMs ability to generate documentation from source code suggest that there is as potential for automating the task of documentation.
\\ \\
Khan and Uddin, \cite{10.1145/3551349.3559548} used a Chat GPT-3 based model called Codex, which is trained on natural language and code, to generate code documentations given source code from the CodeSearchNet dataset, covering six languages (Java, Python, PHP, Go, JavaScript, Ruby). Between the generated documentation and the original documentation from the dataset, several metrics was used to evaluated the result: BLEU score, to compare syntactic similarities, Documentation Length and Flesch-Kincaid Grade Level to compare the Quantity and Readability, TF-IDF score to compare informativeness, and a qualitative comparison by holding two examples up against each other. Showing that code documentation generated by Codex had satisfactory scores in the objective comparisons and even better understandability and extra information were added in the qualitative analysis.
\\ \\
Dvivedi et al. \cite{dvivedi2024comparativeanalysislargelanguage} conducted an analysis on how well LLM models generated code documentation given source code compared to the original code documentation. The model they used were, at the time, the most popular closed- and open-source LLM models, with sizes from 15 billion (Starchat) parameters to 1.76 trillion parameters. They looked at 5 objective metrics: Accuracy, Completeness, Relevance, Understandability and Readability. Concluding that \textit{"with the exception of Starchat, all LLMs demonstrated either equivalent or superior performance compared to the original documentation, highlighting their potential for automating documentation tasks"}.
\subsection{Autonomous Code Documentation Agents}
\label{sec: autonomus code documentaioin agents}
Looking at research on prototypes which tries to integrate code documentation generation to a repository, we see that the use of AI-driven Agents is the typical approach.
\\ \\
Yang et al.\cite{yang2025docagent} propose DocAgent, a multi-agent collaborative system, which generate documentation for a whole repository by leveraging the Chat GPT API. Working in two step: Firstly the Navigator makes a optimal process order, by examine the dependencies in the Repository. Secondly their multi agent system generates documentation for the whole repository.
\\ \\
Using the git pre-commit hook, Luo et al.\cite{luo2024repoagent} have made RepoAGENT which uses Chat GPTs API to created documentation files, which analyses the changes made in the commit, and updates the documentation files. This is an approach very similar to ours, by enabling of documentation in an CI pipeline system like GitHub Actions, provided you have an subscription to the Chat GPT API.
\\ \\
With confidence that various LLM is able to generate function level documentation, see \ref{sec:related work LLM}, this paper will contribute with a proposal to a autonomous code documentation agent prototype which will combine the two approaches mentioned above \ref{sec: autonomus code documentaioin agents} by both offer inline documentation and be integrated into a CI system to be able to keep documentation up to date. But also be in contrast by offering the agent as a free service without the need for a subscription to a cloud AI model, like Chat GPT API.

